\title{Notes on getting relief from a landscape}
\author{
        Simon M. Mudd \\
        School of GeoSciences\\
        University of Edinburgh \\
				Drummond Street \\
				Edinburgh EH8 9XP
}
\date{\today}

\documentclass[12pt]{article}

\begin{document}
\maketitle



\section{Introduction}
At steady state, the equation of the stream power model reduces to:

\begin{equation} \label{eq:profile_eqn_integrate}
z(x) = z(x_b) + \Bigg(\frac{E}{K}\Bigg)^{\frac{1}{n}} \int_{x_b}^{x} \frac{dx}{A(x)^{\frac{m}{n}}},
\end{equation}

where the integration is performed upstream from an arbitrary baselevel ($x_b$) to a chosen point on the river channel, $x$.  The profile is then normalized to a reference drainage area ($A_0$) to ensure the integrand is dimensionless:

\begin{equation} \label{eq:chi_profile}
z(x) = z(x_b) + \Bigg(\frac{U}{K{A_0}^m}\Bigg)^{\frac{1}{n}} \chi,
\end{equation}

The relief, $R$ is then


\begin{equation} \label{eq:relief_profile}
R = z(x_L) - z(x_b) = \Bigg(\frac{U}{K{A_0}^m}\Bigg)^{\frac{1}{n}} \chi_L,
\end{equation}

where $z(x_L)$ is the elevation at the end of the longest channel and $\chi_L$ is the $\chi$ coordinate at the end of the longest channel. 

If we want to calculate the $K$ value needed we simply have:

\begin{equation} \label{eq:K_relief}
K = \frac{U}{{A_0}^m} \Bigg( \frac{R}{\chi_L} \Bigg)^{-n},
\end{equation}


\end{document}

